\documentclass{article}
\setlength\parindent{0pt}
\usepackage{geometry} 
\geometry{letterpaper,left=22.5mm,right=16.1mm,top=48mm,bottom=25mm,headheight=12.5mm, footskip=12.5mm}
\usepackage{fontspec}
\setmainfont{Arial}
\usepackage[spanish,activeacute]{babel}
\usepackage{graphicx}
\usepackage{tikz}
\usetikzlibrary{calc}
\usepackage{anyfontsize}
\usepackage{xcolor,colortbl}
\usepackage{array}
\usepackage{float}
\usepackage{lipsum}
\usepackage{lastpage}
\usepackage{longtable}
\usepackage{multirow}

\definecolor{parte}{rgb}{0.02,0.204,0.404}%5,52,103
\definecolor{nomCur}{rgb}{0.02,0.455,0.773}%5,116,197
\definecolor{fila}{rgb}{0.929,0.929,0.929}%237,237,237
\definecolor{linea}{rgb}{0.749,0.749,0.749}%191,191,191

\newcommand{\codCurso}{EM-XXXX}
\newcommand{\nomCurso}{Introducción a la arepa voladora}
\newcommand{\tipCurso}{Teórico-Práctico}
\newcommand{\elec}{No}
\newcommand{\unAcred}{Ingeniería 50\%, Diseño 50\%}
\newcommand{\ubiPlan}{X Semestre}
\newcommand{\requisito}{EM-XXX Curso X}
\newcommand{\coRequisito}{Ninguno}
\newcommand{\requiDe}{Ninguno}
\newcommand{\asist}{Asistencia Libre}
\newcommand{\sufi}{No tiene suficiencia}
\newcommand{\credito}{4}
\newcommand{\hClass}{4}
\newcommand{\hExtra}{12}
\newcommand{\vigProgra}{X Semestre 20XX}

\newcommand{\nomEscuela}{Escuela de Ingeniería Electromecánica}
\newcommand{\nomPrograma}{Licenciatura en Ingeniería Electromecánica}

\newcommand{\nomProfe}{Pepe Aguilar}
\newcommand{\corProfe}{pagui@itcr.ac.cr}
\newcommand{\consulta}{Consulta: Miércoles 7:30 a.m. – 10:30 a.m.}

\usepackage{fancyhdr}
\pagestyle{fancy}

\lhead{\begin{tikzpicture}[overlay, remember picture]
\node[inner sep=0mm,outer sep=0mm,anchor=north west, %anchor is upper left corner of the graphic
      xshift=18mm, %shifting around
      yshift=-18mm] 
     at (current page.north west) %left upper corner of the page
     {\includegraphics[width=62.5mm]{figuras/Logo.png}}; 
\end{tikzpicture}}
\fancyfoot{}
\lfoot{\textcolor{nomCur}{\nomEscuela - \nomPrograma}}
\fancyfoot[R]{\textcolor{nomCur}{Página \thepage \hspace{1pt} de \pageref{LastPage}}}

\renewcommand{\headrulewidth}{0pt} % elimina la linea en fancy

\begin{document}
\begin{titlepage}
\begin{tikzpicture}[overlay, remember picture]
\node[inner sep=0mm,outer sep=0mm,anchor=north west, %anchor is upper left corner of the graphic
      xshift=4mm, %shifting around
      yshift=-20mm] 
     at (current page.north west) %left upper corner of the page
     {\includegraphics[width=20cm]{figuras/Logo_portada.png}}; 
\end{tikzpicture}

\vspace*{170mm}

\fontsize{14}{0}\selectfont Programa del curso \codCurso

\fontsize{18}{25}\selectfont \textbf{\textcolor{nomCur}{\nomCurso}}

\hspace*{10mm}\fontsize{12}{40}\selectfont \color{black!40!gray}\nomEscuela

\hspace*{10mm}\fontsize{12}{14}\selectfont \color{black!40!gray}\nomPrograma
\end{titlepage}


\fontsize{14}{0}\selectfont\textbf{\textcolor{parte}{I parte: Aspectos relativos al plan de estudios}}

\hspace*{4mm}\fontsize{12}{20}\selectfont\textbf{\textcolor{parte}{1 Datos generales}}
\begin{table}[H]
    \centering
    \begin{tabular}{m{5.7cm}m{10cm}}
        \textbf{Nombre del curso:}  & \nomCurso \\ [5mm]
        \textbf{Código:}  & \codCurso \\ [5mm] 
        \textbf{Tipo de curso:}  & \tipCurso \\ [5mm]
        \textbf{Electivo o no:}  & \elec \\ [5mm]
        \textbf{Nº de créditos:}  & \credito \\ [5mm]
        \textbf{Nº horas de clase por semana:}  & \hClass \\ [5mm]
        \textbf{Nº horas extra clase por semana:} & \hExtra \\ [5mm]
        \textbf{\% de unidades de acreditación:}  & \unAcred \\ [5mm]
        \textbf{Ubicación en el plan de estudios:}  & \ubiPlan \\ [5mm]
        \textbf{Requisitos:}   & \requisito \\ [5mm]
        \textbf{Correquisitos:}  & \coRequisito \\ [5mm]
        \textbf{El curso es requisito de:}  & \requiDe \\ [5mm]
        \textbf{Asistencia:}  & \asist \\ [5mm]
        \textbf{Suficiencia:}  & \sufi \\ [5mm]
        \textbf{Posibilidad de reconocimiento:}  & Es susceptible a reconocimiento automático entre las universidades del CONARE, cuando para esto existe un acuerdo respectivo. \\ [5mm]
        \textbf{Vigencia del programa:}   & \vigProgra
    \end{tabular}
\end{table}

\newpage

\begin{longtable}{p{0.18\textwidth}p{0.72\textwidth}}
        \fontsize{12}{0}\selectfont\textbf{\textcolor{parte}{2 Descripción general}}  & \lipsum[1-2] \\ [5mm]
        \fontsize{12}{0}\selectfont\textbf{\textcolor{parte}{3 Objetivos}}  & \lipsum[1-2] \\
\end{longtable}

{\arrayrulecolor{linea}
\begin{longtable}{m{0.18\textwidth}m{0.24\textwidth}|m{0.24\textwidth}m{0.24\textwidth}}
          \cline{2-4}
          %\rowcolor{fila}
          & Objetivo(s) del curso & Atributo(s) correspondiente(s) & Nivel de desarrollo de cada atributo que se planea alcanzar: Inicial - I, intermedio - M o avanzado - A\\\cline{2-4}
          & \cellcolor{fila}Objetivo 1 & \multirow{2}{*}{TE} & M\\ \cline{2-2} \cline{2-2}
          & Objetivo 2 &  & I\\ \cline{2-2} 
          & \cellcolor{fila}Objetivo 3 & Atributo X & \multirow{3}{*}{A}\\ \cline{2-2} 
          & Objetivo 4 & Atributo Y & \\ \cline{2-2}
          & \cellcolor{fila}Objetivo 5 & TE & \\ \cline{2-4}
\end{longtable}}

\begin{longtable}{p{0.18\textwidth}p{0.72\textwidth}}
        \fontsize{12}{0}\selectfont\textbf{\textcolor{parte}{4 Contenidos}} & \lipsum[1-2] \\ 
\end{longtable}

\section*{\textcolor{parte}{II parte: Aspectos operativos}}
\begin{longtable}{p{0.18\textwidth}p{0.72\textwidth}}
        {\large \textbf{\textcolor{parte}{5 Metodología de enseñanza y aprendizaje}}}  & \lipsum[1-2] \\ [5mm]
        {\large \textbf{\textcolor{parte}{6 Evaluación}}}  & \lipsum[1-2] \\ [5mm] 
        {\large \textbf{\textcolor{parte}{7 Bibliografía}}}  & \lipsum[1-2] \\ [5mm] 
        {\large \textbf{\textcolor{parte}{8 Profesor}}}  & \nomProfe \\
        & \corProfe \\
        & \consulta
\end{longtable}

\end{document}
