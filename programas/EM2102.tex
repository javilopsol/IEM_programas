\documentclass[letterpaper]{article}%
\usepackage{lastpage}%
\usepackage{parskip}%
\usepackage{geometry}%
\geometry{left=22.5mm,right=16.1mm,top=48mm,bottom=25mm,headheight=12.5mm,footskip=12.5mm}%
\usepackage{fontspec}%
\usepackage[spanish,activeacute]{babel}%
\usepackage{graphicx}%
\usepackage{tikz}%
\usepackage{anyfontsize}%
\usepackage{xcolor}%
\usepackage{colortbl}%
\usepackage{array}%
\usepackage{float}%
\usepackage{longtable}%
\usepackage{multirow}%
\usepackage{fancyhdr}%
\usepackage{color}%
\usepackage{ragged2e}%
%
\setmainfont{Arial}%
\usetikzlibrary{calc}%
\definecolor{gris}{rgb}{0.27,0.27,0.27}%
\definecolor{parte}{rgb}{0.02,0.204,0.404}%
\definecolor{azulsuaveTEC}{rgb}{0.02,0.455,0.773}%
\definecolor{fila}{rgb}{0.929,0.929,0.929}%
\definecolor{linea}{rgb}{0.749,0.749,0.749}%
\fancypagestyle{headfoot}{%
\renewcommand{\headrulewidth}{0pt}%
\renewcommand{\footrulewidth}{0pt}%
\fancyhead{%
}%
\fancyfoot{%
}%
\fancyhead[L]{%
\begin{minipage}{0.5\textwidth}%
\flushleft%
\includegraphics[width=62.5mm]{../figuras/Logo.png}%
\end{minipage}%
}%
\fancyfoot[L]{%
\textcolor{azulsuaveTEC}{%
Escuela de Ingeniería Electromecánica {-} Licenciatura en Ingeniería Electromecánica%
}%
}%
\fancyfoot[R]{%
\textcolor{azulsuaveTEC}{%
Página \thepage \hspace{1pt} de \pageref{LastPage}%
}%
}%
}%
%
\begin{document}%
\normalsize%
\thispagestyle{empty}%
\begin{tikzpicture}[overlay,remember picture]%
\node[inner sep = 0mm,outer sep = 0mm,anchor = north west,xshift = -23mm,yshift = 22mm] at (0.0,0.0) {\includegraphics[width=21cm]{../figuras/Logo_portada.png}};%
\end{tikzpicture}%
\vspace*{150mm}%
\par\fontsize{14}{0}\selectfont \textcolor{black}{Programa del curso EM2102}%
\par\fontsize{18}{25}\selectfont \textbf{\textcolor{azulsuaveTEC}{Estática}}%
\par\hspace*{10mm}\fontsize{12}{30}\selectfont \textbf{\textcolor{gris}{Escuela de Ingeniería Electromecánica}}%
\par\hspace*{10mm}\fontsize{12}{14}\selectfont \textbf{\textcolor{gris}{Licenciatura en Ingeniería Electromecánica}}%
\newpage%
\pagestyle{headfoot}%
\par\fontsize{14}{0}\selectfont \textbf{\textcolor{parte}{I parte: Aspectos relativos al plan de estudios}}%
\par\hspace*{4mm}\fontsize{12}{20}\selectfont \textbf{\textcolor{parte}{1 Datos generales}}%
\renewcommand{\arraystretch}{1.5}%
\begin{longtable}{m{7cm}m{9cm}}%
\textbf{Nombre del curso:}&Estática\\%
\textbf{Código:}&EM2102\\%
\textbf{Tipo de curso:}&Teórico\\%
\textbf{Electivo o no:}&No\\%
\textbf{Nº de créditos:}&3\\%
\textbf{Nº horas de clase por semana:}&4\\%
\textbf{Nº horas extraclase por semana:}&5\\%
\textbf{\% de areas curriculares:}&6.0\\%
\textbf{Ubicación en el plan de estudios:}&3\\%
\textbf{Requisitos:}&nan\\%
\textbf{Correquisitos:}&nan\\%
\textbf{El curso es requisito de:}&nan\\%
\textbf{Asistencia:}&nan\\%
\textbf{Suficiencia:}&nan\\%
\textbf{Posibilidad de reconocimiento:}&nan\\%
\textbf{Vigencia del programa:}&nan\\%
\end{longtable}%
\newpage%
\renewcommand{\arraystretch}{1.5}%
\begin{longtable}{p{0.18\textwidth}p{0.72\textwidth}}%
\par\fontsize{12}{0}\selectfont \textbf{\textcolor{parte}{2 Descripción general}}&El curso de dibujo técnico contribuye significativamente a la formación y desarrollo profesional de los estudiantes, equipándolos con las habilidades y herramientas necesarias para la comunicación, el diseño y la ejecución en el ámbito de la ingeniería. Además, fomenta tanto el trabajo en equipo como las habilidades individuales desde un nivel inicial, como parte de las habilidades blandas que se promueven.\newline%
Entre los aprendizajes más destacados se encuentran la capacidad para interpretar y aplicar las normas INTE/ISO de dibujo técnico en situaciones prácticas, así como la habilidad para visualizar y representar objetos tridimensionales en un plano bidimensional y viceversa. Asimismo, los estudiantes desarrollan una destreza notable en el uso de herramientas CAD para la elaboración de planos. Además de las habilidades técnicas, se busca promover el compromiso, el respeto y la ética profesional entre los participantes.\newline%
Este curso sienta las bases fundamentales para asignaturas más avanzadas en el campo del diseño y la ingeniería mecánica. Proporciona una comprensión sólida de los principios y técnicas de dibujo técnico que son esenciales para el desarrollo de proyectos más complejos en áreas como el diseño de máquinas, la ingeniería de manufactura y la ingeniería de productos. De esta manera, establece una conexión trascendente con otros cursos de la carrera, preparando a los estudiantes para enfrentar desafíos en su trayectoria académica y profesional.\\%
\par\fontsize{12}{0}\selectfont \textbf{\textcolor{parte}{2 Objetivos}}&Al final del curso el estudiante será capaz de elaborar un plano de una pieza mecánica, según las normas INTE/ISO. Dicho plano debe contener la información necesaria y suficiente para la interpretación de la forma y dimensiones de la pieza. Además, el estudiante será capaz de interpretar correctamente la información contenida en un plano mecánico, realizado de acuerdo con las normas INTE/ISO.\\%
&1.	Dar a conocer al estudiante las normas INTE/ISO y similares aplicadas al Dibujo Técnico. \newline%
2.	Dominar el uso de tablas de la norma INTE/ISO en el Dibujo (Rotulado, líneas, sólidos, cortes, etc.) \newline%
3.	Representar mediante proyecciones ortogonales y proyecciones axonométricas una pieza. \newline%
4.	Representar mediante vistas y cortes una pieza. \newline%
 5.	Aplicar los criterios para acotar una pieza en las proyecciones ortogonales. \\%
\end{longtable}%
\end{document}