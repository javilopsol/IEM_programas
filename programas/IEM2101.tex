\documentclass[letterpaper]{article}%
\usepackage{lastpage}%
\usepackage{parskip}%
\usepackage{geometry}%
\geometry{left=22.5mm,right=16.1mm,top=48mm,bottom=25mm,headheight=12.5mm,footskip=12.5mm}%
\usepackage{fontspec}%
\usepackage[spanish,activeacute]{babel}%
\usepackage{anyfontsize}%
\usepackage{fancyhdr}%
\usepackage{csquotes}%
\usepackage[style=ieee,backend=biber]{biblatex}%
\usepackage{color}%
\usepackage{tikz}%
\usepackage{tabularx}%
\usepackage{longtable}%
\usepackage{ltablex}%
\usepackage{ragged2e}%
\usepackage{graphicx}%
\usepackage{xcolor}%
%
\setmainfont{Arial}%
\addbibresource{../bibliografia.bib}%
\renewcommand*{\bibfont}{\fontsize{12}{16}\selectfont}%
\definecolor{gris}{rgb}{0.27,0.27,0.27}%
\definecolor{parte}{rgb}{0.02,0.204,0.404}%
\definecolor{azulsuaveTEC}{rgb}{0.02,0.455,0.773}%
\definecolor{fila}{rgb}{0.929,0.929,0.929}%
\definecolor{linea}{rgb}{0.749,0.749,0.749}%
\fancypagestyle{headfoot}{%
\renewcommand{\headrulewidth}{0pt}%
\renewcommand{\footrulewidth}{0pt}%
\fancyhead{%
}%
\fancyfoot{%
}%
\fancyhead[L]{%
\begin{minipage}{0.5\textwidth}%
\flushleft%
\includegraphics[width=62.5mm]{../figuras/Logo.png}%
\end{minipage}%
}%
\fancyfoot[L]{%
\textcolor{azulsuaveTEC}{%
Escuela de Ingeniería Electromecánica%
}%
\par \parbox{0.85\textwidth}{%
\fontsize{8}{0}\selectfont \textcolor{azulsuaveTEC}{Carreras de: Ingeniería en Mantenimiento Industrial; Ingeniería en Electrónica; Ingeniería en Producción Industrial; Ingeniería Mecatrónica e Ingeniería en Materiales}%
}%
}%
\fancyfoot[R]{%
\textcolor{azulsuaveTEC}{%
Página \thepage \hspace{1pt} de \pageref*{LastPage}%
}%
}%
}%
%
\begin{document}%
\normalsize%
\thispagestyle{empty}%
\begin{tikzpicture}[overlay,remember picture]%
\node[inner sep = 0mm,outer sep = 0mm,anchor = north west,xshift = -23mm,yshift = 22mm] at (0.0,0.0) {\includegraphics[width=21cm]{../figuras/Logo_portada.png}};%
\end{tikzpicture}%
\vspace*{150mm}%
\par\fontsize{14}{0}\selectfont \textcolor{black}{Programa del curso IEM2101}%
\par\fontsize{18}{25}\selectfont \textbf{\textcolor{azulsuaveTEC}{Dibujo técnico asistido por computadora}}%
\begin{tabularx}{\textwidth}{m{0.02\textwidth}m{0.98\textwidth}}%
&\hspace*{0mm}\fontsize{12}{14}\selectfont \textbf{\textcolor{gris}{Escuela de Ingeniería Electromecánica}}\\%
[-4pt]%
&\hspace*{0mm}\fontsize{12}{14}\selectfont \textbf{\textcolor{gris}{Carreras de: Ingeniería en Mantenimiento Industrial; Ingeniería en Electrónica; Ingeniería en Producción Industrial; Ingeniería Mecatrónica e Ingeniería en Materiales}}\\%
\end{tabularx}%
\newpage%
\pagestyle{headfoot}%
\par\fontsize{14}{0}\selectfont \textbf{\textcolor{parte}{I parte: Aspectos relativos al plan de estudios}}%
\par\hspace*{4mm}\fontsize{12}{20}\selectfont \textbf{\textcolor{parte}{1 Datos generales}}%
\begin{tabularx}{\textwidth}{p{7cm}p{9cm}}%
\textbf{Nombre del curso:}&Dibujo técnico asistido por computadora\\%
[10pt]%
\textbf{Código:}&IEM2101\\%
[10pt]%
\textbf{Tipo de curso:}&Teórico\\%
[10pt]%
\textbf{Electivo o no:}&No\\%
[10pt]%
\textbf{Nº de créditos:}&3\\%
[10pt]%
\textbf{Nº horas de clase por semana:}&4\\%
[10pt]%
\textbf{Nº horas extraclase por semana:}&5\\%
[10pt]%
\textbf{\% de areas curriculares:}&6.0\% del area: \textbf{mecánica}\\%
[10pt]%
\textbf{Ubicación en el plan de estudios:}&Curso de 1
 semestre en Ingeniería en Electrónica e Ingeniería en Producción Industrial. Curso de 2
 semestre en Ingeniería en Mantenimiento Industrial. Curso de 3
 semestre en Ingeniería Mecatrónica. Curso de 4\textsuperscript{to} semestre en Ingeniería en Materiales. \\%
[10pt]%
\textbf{Requisitos:}&nan\\%
[10pt]%
\textbf{Correquisitos:}&nan\\%
[10pt]%
\textbf{El curso es requisito de:}&nan\\%
[10pt]%
\textbf{Asistencia:}&nan\\%
[10pt]%
\textbf{Suficiencia:}&nan\\%
[10pt]%
\textbf{Posibilidad de reconocimiento:}&nan\\%
[10pt]%
\textbf{Vigencia del programa:}&nan\\%
[10pt]%
\end{tabularx}%
\begin{tabularx}{\textwidth}{>{\raggedright}p{0.18\textwidth}p{0.72\textwidth}}%
\par\fontsize{12}{0}\selectfont \textbf{\textcolor{parte}{2 Descripción general}}&El curso de Dibujo Técnico Asistido por Computadora contribuye significativamente a la formación y desarrollo profesional de los estudiantes, equipándolos con las habilidades y herramientas necesarias para la comunicación, el diseño y la ejecución en el ámbito de la ingeniería.
\newline%
Entre los aprendizajes más destacados se encuentran:  interpretar y aplicar las normas INTE/ISO de dibujo técnico en situaciones prácticas; visualizar y representar objetos tridimensionales en un plano bidimensional y viceversa; desarrollar destrezas en el uso de herramientas CAD para la elaboración de planos; y además de las habilidades técnicas, se busca promover el compromiso, el respeto y la ética profesional entre los participantes.​
\newline%
Este curso: sienta las bases fundamentales para asignaturas más avanzadas en el campo del diseño y la ingeniería mecánica. Proporciona una comprensión sólida de los principios y técnicas de dibujo técnico que son esenciales para el desarrollo de proyectos más complejos las áreas de diseño y manufactura. De esta manera, establece una conexión con otros cursos de la carrera, preparando a los estudiantes para enfrentar desafíos en su trayectoria académica y profesional.​
\newline%
En caso de que un estudiante requiera apoyos educativos, podrá solicitarlos a través del Departamento de Orientación y Psicología.\\%
\end{tabularx}%
\begin{tabularx}{\textwidth}{>{\raggedright}p{0.18\textwidth}p{0.72\textwidth}}%
\par\fontsize{12}{0}\selectfont \textbf{\textcolor{parte}{3 Objetivos}}&Al final del curso la persona estudiante será capaz de al final del curso el estudiante será capaz de elaborar un plano de una pieza mecánica, según las normas INTE/ISO. Dicho plano debe contener la información necesaria y suficiente para la interpretación de la forma y dimensiones de la pieza. Además, el estudiante será capaz de interpretar correctamente la información contenida en un plano mecánico, realizado de acuerdo con las normas INTE/ISO.\\%
\end{tabularx}%
\begin{tabularx}{\textwidth}{>{\raggedleft}p{0.18\textwidth}p{0.72\textwidth}}%
&La persona estudiante será capaz de:\\%
\textbullet&Dar a conocer al estudiante las normas INTE/ISO y similares aplicadas al Dibujo Técnico. 
\\%
\textbullet&Dominar el uso de tablas de la norma INTE/ISO en el Dibujo (Rotulado, líneas, sólidos, cortes, etc.) 
\\%
\textbullet&Representar mediante proyecciones ortogonales y proyecciones axonométricas una pieza. 
\\%
\textbullet&Representar mediante vistas y cortes una pieza. 
\\%
\textbullet& Aplicar los criterios para acotar una pieza en las proyecciones ortogonales. \\%
\end{tabularx}%
\renewcommand{\arraystretch}{1.5}%
\begin{longtable}{>{\raggedright}p{0.18\textwidth}p{0.72\textwidth}}%
\par\fontsize{12}{0}\selectfont \textbf{\textcolor{parte}{4 Contenidos}}&1. Generalidades\\%
&\hspace{0.02\linewidth}\parbox{0.98\linewidth}{1.1. El Dibujo Técnico como lenguaje. Historia del Dibujo Técnico.}\\%
&\hspace{0.02\linewidth}\parbox{0.98\linewidth}{1.2. Objetivos del curso}\\%
&\hspace{0.02\linewidth}\parbox{0.98\linewidth}{1.3. Instrumentos de Dibujo: Escalímetro, escuadras.}\\%
&\hspace{0.02\linewidth}\parbox{0.98\linewidth}{1.4. Rotulado técnico y Formatos para dibujo Técnico}\\%
&\hspace{0.02\linewidth}\parbox{0.98\linewidth}{1.5. Norma de rotulado INTE-ISO 3098/0/2/3-2008}\\%
&\hspace{0.02\linewidth}\parbox{0.98\linewidth}{1.6. Formatos según INTE-ISO 5457-2008}\\%
&\hspace{0.02\linewidth}\parbox{0.98\linewidth}{1.7. Información que debe contener un cajetín INTE-ISO 7200-2008}\\%
&2. Geometría Descriptiva (2 hrs)\\%
&\hspace{0.02\linewidth}\parbox{0.98\linewidth}{2.1. Consideraciones fundamentales de la Geometría Descriptiva :}\\%
&\hspace{0.04\linewidth}\parbox{0.96\linewidth}{2.1.1. Objetivos del curso}\\%
&\hspace{0.04\linewidth}\parbox{0.96\linewidth}{2.1.2. Concepto de Geometría Descriptiva y su Historia}\\%
&\hspace{0.02\linewidth}\parbox{0.98\linewidth}{2.2. Proyección del punto, el segmento y los planos en el espacio}\\%
&\hspace{0.04\linewidth}\parbox{0.96\linewidth}{2.2.1. Proyección de un punto, el segmento y los planos en las vistas}\\%
&\hspace{0.04\linewidth}\parbox{0.96\linewidth}{2.2.2. Proyección de un punto, el segmento y los planos en el espacio}\\%
&\hspace{0.02\linewidth}\parbox{0.98\linewidth}{2.3. Longitudes y dimensiones naturales}\\%
&\hspace{0.04\linewidth}\parbox{0.96\linewidth}{2.3.1. Procedimiento de rotación para encontrar la dimensión real de un segmento y un plano;}\\%
&\hspace{0.04\linewidth}\parbox{0.96\linewidth}{2.3.2. Procedimiento de sustitución de planos para encontrar la dimensión real de un segmento y un plano;}\\%
&\hspace{0.04\linewidth}\parbox{0.96\linewidth}{2.3.3. Procedimiento de superposición para encontrar la dimensión real de un segmento y un plano.}\\%
&\hspace{0.02\linewidth}\parbox{0.98\linewidth}{2.4. Características particulares de la representación de los Cuerpos Geométricos}\\%
&\hspace{0.04\linewidth}\parbox{0.96\linewidth}{2.4.1. Características particulares del prisma y su representación;}\\%
&\hspace{0.04\linewidth}\parbox{0.96\linewidth}{2.4.2. Características del cono y su representación;}\\%
&\hspace{0.04\linewidth}\parbox{0.96\linewidth}{2.4.3. Características de la pirámide y su representación;}\\%
&\hspace{0.04\linewidth}\parbox{0.96\linewidth}{2.4.4. Características del cilindro y su representación;}\\%
&\hspace{0.04\linewidth}\parbox{0.96\linewidth}{2.4.5. Características del toroide y su representación;}\\%
&\hspace{0.04\linewidth}\parbox{0.96\linewidth}{2.4.6. Características de la esfera y su representación}\\%
&3. Escalas (4 hrs)\\%
&\hspace{0.02\linewidth}\parbox{0.98\linewidth}{3.1. Concepto de escalas}\\%
&\hspace{0.02\linewidth}\parbox{0.98\linewidth}{3.2. Escalas según INTE ISO 5455-2008}\\%
&4. Proyecciones ortogonales (14 hrs)\\%
&\hspace{0.02\linewidth}\parbox{0.98\linewidth}{4.1. Sistema de proyección y designación de vistas según la norma INTE- ISO 128/30/34-2008}\\%
&\hspace{0.02\linewidth}\parbox{0.98\linewidth}{4.2. Criterios de selección de la vista frontal y la ubicación de las otras vistas.}\\%
&\hspace{0.02\linewidth}\parbox{0.98\linewidth}{4.3. Cantidad de vistas que definen un objeto.}\\%
&\hspace{0.02\linewidth}\parbox{0.98\linewidth}{4.4. Significado y utilización de los tipos de líneas.}\\%
&\hspace{0.02\linewidth}\parbox{0.98\linewidth}{4.5. Tipos de líneas según la norma INTE ISO 128/20/21/22/23/24-2008}\\%
&5. Presentación de un plano:\\%
&\hspace{0.02\linewidth}\parbox{0.98\linewidth}{5.1. Calidad de líneas}\\%
&\hspace{0.02\linewidth}\parbox{0.98\linewidth}{5.2. Orden y adecuada ubicación de la información}\\%
&\hspace{0.02\linewidth}\parbox{0.98\linewidth}{5.3. Especificaciones técnicas.}\\%
&\hspace{0.02\linewidth}\parbox{0.98\linewidth}{5.4. Práctica de proyecciones ortogonales croquizando “a mano alzada”.}\\%
&\hspace{0.02\linewidth}\parbox{0.98\linewidth}{5.5. Práctica de proyecciones ortogonales utilizando Software de dibujo.}\\%
&6. Proyecciones Axonométricas (12 hrs)\\%
&\hspace{0.02\linewidth}\parbox{0.98\linewidth}{6.1. Ejes de proyección}\\%
&\hspace{0.02\linewidth}\parbox{0.98\linewidth}{6.2. Tipos de axonometrías}\\%
&\hspace{0.02\linewidth}\parbox{0.98\linewidth}{6.3. Proyecciones axonométricas a mano alzada y utilizando software de dibujo.}\\%
&Acotado (4 hrs) \\%
&\hspace{0.02\linewidth}\parbox{0.98\linewidth}{6.4. Normas y recomendaciones INTE/ISO 129/1-2008 sobre acotado.}\\%
&\hspace{0.02\linewidth}\parbox{0.98\linewidth}{6.5. Líneas utilizadas en el acotado}\\%
&\hspace{0.02\linewidth}\parbox{0.98\linewidth}{6.6. Posición de la cota}\\%
&\hspace{0.02\linewidth}\parbox{0.98\linewidth}{6.7. Rotulado de cotas}\\%
&\hspace{0.02\linewidth}\parbox{0.98\linewidth}{6.8. Criterios para la acotación correcta de piezas.}\\%
&\hspace{0.02\linewidth}\parbox{0.98\linewidth}{6.9. Relación entre cota y escala.}\\%
&7. Cortes y secciones (4 hrs)\\%
&\hspace{0.02\linewidth}\parbox{0.98\linewidth}{7.1. Concepto de cortes y secciones. Conveniencia de su utilización}\\%
&\hspace{0.02\linewidth}\parbox{0.98\linewidth}{7.2. Representación e indicación de cortes según la norma INTE/ISO 128/40-44- 2008}\\%
&\hspace{0.02\linewidth}\parbox{0.98\linewidth}{7.3. Achurado}\\%
&\hspace{0.02\linewidth}\parbox{0.98\linewidth}{7.4. Tipos de cortes.}\\%
&\hspace{0.04\linewidth}\parbox{0.96\linewidth}{7.4.1. Sección en un plano de corte.}\\%
&\hspace{0.04\linewidth}\parbox{0.96\linewidth}{7.4.2. Sección en dos planos paralelos.}\\%
&\hspace{0.04\linewidth}\parbox{0.96\linewidth}{7.4.3. Sección en tres planos de corte continuos}\\%
&\hspace{0.04\linewidth}\parbox{0.96\linewidth}{7.4.4. Sección en dos planos de intersección}\\%
&\hspace{0.04\linewidth}\parbox{0.96\linewidth}{7.4.5. Plano de corte posesionado parcialmente fuera de la pieza}\\%
&\hspace{0.04\linewidth}\parbox{0.96\linewidth}{7.4.6. Sección removida de una vista}\\%
&\hspace{0.04\linewidth}\parbox{0.96\linewidth}{7.4.7. Secciones sucesivas}\\%
&\hspace{0.04\linewidth}\parbox{0.96\linewidth}{7.4.8. Cortes oblicuos o auxiliares}\\%
&\hspace{0.04\linewidth}\parbox{0.96\linewidth}{7.4.9. Cortes y secciones parciales}\\%
&\hspace{0.04\linewidth}\parbox{0.96\linewidth}{7.4.10. Corte y secciones de piezas simétricas}\\%
&8. Vistas auxiliares simples (4 hrs)\\%
&\hspace{0.02\linewidth}\parbox{0.98\linewidth}{8.1. Vistas auxiliares}\\%
&\hspace{0.02\linewidth}\parbox{0.98\linewidth}{8.2. Tipos de vistas auxiliares}\\%
&\hspace{0.02\linewidth}\parbox{0.98\linewidth}{8.3. Ubicación de las vistas auxiliares}\\%
&\hspace{0.02\linewidth}\parbox{0.98\linewidth}{8.4. Rotulado de vistas auxiliares.}\\%
\end{longtable}%
\par\fontsize{14}{0}\selectfont \textbf{\textcolor{parte}{II parte: Aspectos operativos}}%
\par\fontsize{12}{0}\selectfont \textbf{\textcolor{parte}{ }}%
\renewcommand{\arraystretch}{1.5}%
\begin{tabularx}{\textwidth}{>{\raggedright}p{0.18\textwidth}p{0.72\textwidth}}%
\par\fontsize{12}{0}\selectfont \textbf{\textcolor{parte}{5 Metodología de enseñanza y aprendizaje}}&Modalidad presencial en el horario de clase correspondiente. En las sesiones se expondrán los conceptos teóricos relevantes de cada tema por medio de material escrito o audiovisual, los cuales serán trabajados por el estudiante en numerosas prácticas realizadas en clases y tareas coordinadas.\\%
\end{tabularx}%
\begin{tabularx}{\textwidth}{>{\raggedright}m{0.18\textwidth}m{0.07\textwidth}m{0.17\textwidth}m{0.17\textwidth}m{0.17\textwidth}m{0.04\textwidth}}%
\cline{3%
-%
5}%
\par\fontsize{12}{0}\selectfont \textbf{\textcolor{parte}{6 Evaluación}}&&\par\fontsize{12}{16}\selectfont \textbf{\textcolor{black}{Tipo}}&\par\fontsize{12}{16}\selectfont \textbf{\textcolor{black}{Cantidad}}&\par\fontsize{12}{16}\selectfont \textbf{\textcolor{black}{Porcentaje}}&\\%
[12pt]%
\cline{3%
-%
5}%
&&
&None&None&\\%
[12pt]%
\cline{3%
-%
5}%
&&Examen Parcial&1&25
&\\%
[12pt]%
\cline{3%
-%
5}%
&&Examen Parcial&1&35
&\\%
[12pt]%
\cline{3%
-%
5}%
&&Proyecto grupal&1&25
&\\%
[12pt]%
\cline{3%
-%
5}%
&&Tareas digitales coordinadas&1&15	 	 	 	 &\\%
[12pt]%
\end{tabularx}%
\renewcommand{\arraystretch}{1.5}%
\begin{longtable}{>{\raggedright}p{0.18\textwidth}p{0.72\textwidth}}%
\par\fontsize{12}{0}\selectfont \textbf{\textcolor{parte}{7 Bibliografía}}&
\nocite{hernandez2013dibujo}
\nocite{lieu2011dibujo}
\nocite{bogoliubov1988dibujo}
\nocite{bogoliubov1989tareas}
\printbibliography[heading=none]\\%
\end{longtable}%
\renewcommand{\arraystretch}{1.5}%
\begin{longtable}{>{\raggedright}p{0.18\textwidth}p{0.72\textwidth}}%
\par\fontsize{12}{0}\selectfont \textbf{\textcolor{parte}{8 Profesor}}&profesor\\%
\end{longtable}%
\end{document}