\documentclass[letterpaper]{article}%
\usepackage{lastpage}%
\usepackage{parskip}%
\usepackage{geometry}%
\geometry{left=22.5mm,right=16.1mm,top=48mm,bottom=25mm,headheight=12.5mm,footskip=12.5mm}%
\usepackage{fontspec}%
\usepackage[spanish,activeacute]{babel}%
\usepackage{graphicx}%
\usepackage{tikz}%
\usepackage{anyfontsize}%
\usepackage{xcolor}%
\usepackage{colortbl}%
\usepackage{array}%
\usepackage{float}%
\usepackage{longtable}%
\usepackage{multirow}%
\usepackage{fancyhdr}%
\usepackage{color}%
\usepackage{ragged2e}%
\newcommand{\nomEscue}{Escuela de Ingeniería Electromecánica}%
\newcommand{\nomProgr}{Ingeniería en Mantenimiento Industrial}%
\newcommand{\codCurso}{MI2207}%
\newcommand{\nomCurso}{Electricidad II}%
\newcommand{\tipCurso}{Teórico}%
\newcommand{\eleCurso}{No}%
\newcommand{\porAreas}{100\% Ciencias de la Ingeniería}%
\newcommand{\ubiPlane}{Curso del 5to semestre}%
\newcommand{\susRequi}{CA{-}2125 Elementos de Computación
\newline%
MI{-}2205 Electricidad I}%
\newcommand{\corRequi}{MI{-}2204 Laboratorio de Electricidad II}%
\newcommand{\essRequi}{MI{-}3209 Electricidad III}%
\newcommand{\tipAsist}{Libre}%
\newcommand{\posRecon}{Si}%
\newcommand{\posSufic}{Si}%
\newcommand{\numCredi}{4}%
\newcommand{\horClass}{4}%
\newcommand{\horExtra}{12}%
\newcommand{\vigProgr}{I semestre de 2024}%
\newcommand{\nomProfe}{Juan José Rojas Hernández}%
\newcommand{\corProfe}{juan.rojas@itcr.ac.cr}%
\newcommand{\conProfe}{Miercoles 7:30 a.m. {-} 10: 30 a.m.}%
\setmainfont{Arial}%
\usetikzlibrary{calc}%
\definecolor{gris}{rgb}{0.27,0.27,0.27}%
\definecolor{parte}{rgb}{0.02,0.204,0.404}%
\definecolor{azulsuaveTEC}{rgb}{0.02,0.455,0.773}%
\definecolor{fila}{rgb}{0.929,0.929,0.929}%
\definecolor{linea}{rgb}{0.749,0.749,0.749}%
\fancypagestyle{headfoot}{%
\renewcommand{\headrulewidth}{0pt}%
\renewcommand{\footrulewidth}{0pt}%
\fancyhead{%
}%
\fancyfoot{%
}%
\fancyhead[L]{%
\begin{minipage}{0.5\textwidth}%
\flushleft%
\includegraphics[width=62.5mm]{../figuras/Logo.png}%
\end{minipage}%
}%
\fancyfoot[L]{%
\textcolor{azulsuaveTEC}{%
Escuela de Ingeniería Electromecánica {-} Ingeniería en Mantenimiento Industrial%
}%
}%
\fancyfoot[R]{%
\textcolor{azulsuaveTEC}{%
Página \thepage \hspace{1pt} de \pageref{LastPage}%
}%
}%
}%
%
\begin{document}%
\normalsize%
\thispagestyle{empty}%
\begin{tikzpicture}[overlay,remember picture]%
\node[inner sep = 0mm,outer sep = 0mm,anchor = north west,xshift = -23mm,yshift = 22mm] at (0.0,0.0) {\includegraphics[width=21cm]{../figuras/Logo_portada.png}};%
\end{tikzpicture}%
\vspace*{150mm}%
\par\fontsize{14}{0}\selectfont \textcolor{black}{Programa del curso MI2207}%
\par\fontsize{18}{25}\selectfont \textbf{\textcolor{azulsuaveTEC}{Electricidad II}}%
\par\hspace*{10mm}\fontsize{12}{30}\selectfont \textbf{\textcolor{gris}{Escuela de Ingeniería Electromecánica}}%
\par\hspace*{10mm}\fontsize{12}{14}\selectfont \textbf{\textcolor{gris}{Ingeniería en Mantenimiento Industrial}}%
\newpage%
\pagestyle{headfoot}%
\par\fontsize{14}{0}\selectfont \textbf{\textcolor{parte}{I parte: Aspectos relativos al plan de estudios}}%
\par\hspace*{4mm}\fontsize{12}{20}\selectfont \textbf{\textcolor{parte}{1 Datos generales}}%
\renewcommand{\arraystretch}{1.5}%
\begin{longtable}{m{7cm}m{9cm}}%
\textbf{Nombre del curso:}&Electricidad II\\%
\textbf{Código:}&MI2207\\%
\textbf{Tipo de curso:}&Teórico\\%
\textbf{Electivo o no:}&No\\%
\textbf{Nº de créditos:}&4\\%
\textbf{Nº horas de clase por semana:}&4\\%
\textbf{Nº horas extraclase por semana:}&12\\%
\textbf{\% de areas curriculares:}&100\% Ciencias de la Ingeniería\\%
\textbf{Ubicación en el plan de estudios:}&Curso del 5to semestre\\%
\textbf{Requisitos:}&CA{-}2125 Elementos de Computación
\newline%
MI{-}2205 Electricidad I\\%
\textbf{Correquisitos:}&MI{-}2204 Laboratorio de Electricidad II\\%
\textbf{El curso es requisito de:}&MI{-}3209 Electricidad III\\%
\textbf{Asistencia:}&Libre\\%
\textbf{Suficiencia:}&Si\\%
\textbf{Posibilidad de reconocimiento:}&Si\\%
\textbf{Vigencia del programa:}&I semestre de 2024\\%
\end{longtable}%
\newpage%
\renewcommand{\arraystretch}{1.5}%
\begin{longtable}{p{0.18\textwidth}p{0.72\textwidth}}%
\par\fontsize{12}{0}\selectfont \textbf{\textcolor{parte}{2 Descripción general}}&Este curso profundiza el estudio de señales sinusoidales en circuitos lineales RLC brindando herramientas de análisis en el dominio del tiempo, respuesta en frecuencia y electromagnetismo. Además, enfoca los temas de circuitos monofásicos y trifásicos, cálculo de potencias y teoría de transformadores al área de redes de distribución y transmisión. Su aprovechamiento complementa las bases en la teoría general de circuitos eléctricos y aporta conocimientos necesarios para abordar apropiadamente los cursos posteriores de Electricidad III (MI{-}3209), Máquinas Eléctricas (MI{-}3210) y Sistemas de Potencia (MI{-}5215).\\%
\par\fontsize{12}{0}\selectfont \textbf{\textcolor{parte}{2 Objetivos}}&Brindar al estudiante las bases teóricas que describen los circuitos de corriente alterna y
\newline%
abordar herramientas computacionales para análisis y simulación.\\%
&1. Comprender el comportamiento de los distintos elementos que conforman un circuito RLC bajo excitación sinusoidal.
\newline%
     2. Utilizar herramientas analíticas para el estudio de circuitos en el dominio del tiempo y frecuencia, resonancia de circuitos RLC y teoría electromagnética.
\newline%
     3. Analizar el comportamiento de circuitos polifásicos bajo distintas cargas, enfocando su estudio a redes de distribución eléctrica. 
\newline%
    4. Analizar circuitos con transformadores y autotransformadores como elementos básicos de máquinas eléctricas.\\%
\end{longtable}%
\end{document}